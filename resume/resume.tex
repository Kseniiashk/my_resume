\documentclass[10pt, a4paper]{altacv}
\usepackage[T2A]{fontenc}
\usepackage[utf8]{inputenc}
\usepackage[russian]{babel}
\usepackage{hyperref}
\usepackage{fontspec}
\usepackage{xcolor}
\usepackage{geometry}

% Надежная настройка шрифта
\setmainfont{DejaVu Sans}[
  Path = /usr/share/fonts/,
  Extension = .ttf,
  UprightFont = *,
  BoldFont = *-Bold,
  ItalicFont = *-Oblique,
  BoldItalicFont = *-BoldOblique
]

% Настройки геометрии
\geometry{a4paper, left=1cm, right=1cm, top=1cm, bottom=1cm}

\renewcommand{\divider}{\textcolor{body!30}{\hspace{1pt}} % Добавлен цвет разделителя
\definecolor{VividPurple}{HTML}{000000}
\definecolor{SlateGrey}{HTML}{2E2E2E}
\definecolor{LightGrey}{HTML}{666666}
\colorlet{heading}{VividPurple}
\colorlet{accent}{VividPurple}
\colorlet{emphasis}{SlateGrey}
\colorlet{body}{LightGrey}

% Настройка гиперссылок
\hypersetup{
    colorlinks=true,
    linkcolor=accent,
    urlcolor=accent,
    pdftitle={Резюме Ксении Шкулевой},
    pdfauthor={Ксения Шкулева}
}

\begin{document}
\name{Ксения Шкулева}
\tagline{Data Scientist • Специалист по ИИ}
\personalinfo{
  \email{ksshkuleva@gmail.com}
  \phone{+7 (965) 178-5997}
  \location{Москва, Россия}
  \homepage{github.com/Kseniiashk}
  \linkedin{linkedin.com/in/ksenia-shkuleva} % Обновленная ссылка
  \github{github.com/Kseniiashk} % Добавлен GitHub
  \orcid{orcid.org/0000-0000-0000-0000} % При наличии
}

\begin{fullwidth}
\makecvheader
\end{fullwidth}

% Образование
\cvsection{Образование}
\cvevent{Бакалавр компьютерных наук}{НИУ ВШЭ}{2023 -- настоящее время}{Москва}
\begin{itemize}
  \item Специализация: Искусственный интеллект и машинное обучение
  \item Участник школы по анализу данных НИУ ВШЭ
  \item Курсы: Алгоритмы и структуры данных, Машинное обучение, NLP
\end{itemize}

\divider

\cvevent{Информатический класс}{СУНЦ МГУ}{2022 -- 2024}{Москва}
\begin{itemize}
  \item Углубленное изучение программирования и математики
  \item Подготовка к олимпиадам по информатике
\end{itemize}

% Опыт работы
\cvsection{Опыт работы}
\cvevent{Стажер RAG Developer}{T-Bank}{Март 2024 -- настоящее время}{Москва}
\begin{itemize}
  \item Разработка RAG-пайплайнов с использованием NLP
  \item Реализация гибридного поиска документов (сокращение времени обработки на 40\%)
  \item Интеграция с чат-ботами через FastAPI
  \item Оптимизация промптов для LLM (улучшение релевантности ответов на 30\%)
  \item Контейнеризация решений с Docker
\end{itemize}

\divider

\cvevent{Стажер ML Developer}{T-Bank}{Июнь 2024 -- Август 2024}{Москва}
\begin{itemize}
  \item Разработка Telegram-бота для автоматизации обработки запросов (500+ в день)
  \item Создание системы кластеризации вопросов (точность 92\%)
  \item Анализ тематик вопросов пользователей (выявление 5 ключевых проблем)
  \item Предобработка текстовых данных и подбор моделей
\end{itemize}

\divider

\cvevent{PHP-программист}{DoorHan}{Июнь 2023 -- Март 2024}{Москва}
\begin{itemize}
  \item Разработка и поддержка веб-приложений
  \item Создание новых функций для корпоративной CRM-системы
  \item Оптимизация SQL-запросов (сокращение времени отклика на 35\%)
  \item Работа с MySQL, JavaScript, CSS/HTML
\end{itemize}

% Навыки
\cvsection{Технические навыки}
\begin{cvskills}
  \cvskill{Языки}{Python, SQL, JavaScript, PHP, C++}
  \cvskill{ML/DS}{TensorFlow, PyTorch, NLP, RAG-системы, Кластеризация}
  \cvskill{Инструменты}{Docker, Git, FastAPI, Jupyter, MySQL, Linux}
  \cvskill{Концепции}{Алгоритмы, MLOps, Асинхронное программирование}
\end{cvskills}

% Достижения
\cvsection{Достижения}
\begin{itemize}
  \item \textbf{Победитель} Всероссийской олимпиады по информатике (2024)
  \item \textbf{Призер} Всероссийской олимпиады по информатике (2023)
  \item \textbf{Финалист} Всероссийской олимпиады по ИИ (2023)
  \item \textbf{Призер} Открытой олимпиады по программированию (2023, 2024)
  \item \textbf{Призер} Всероссийской командной школьной олимпиады (2022, 2023)
\end{itemize}

% Проекты
\cvsection{Проекты}
\cvproject{Telegram-бот для генерации изображений}
\begin{itemize}
  \item Разработка GAN-модели для генерации изображений
  \item Интеграция с Telegram через асинхронные запросы
  \item Модель классификации эмоций животных (EfficientNetB5)
\end{itemize}

\divider

\cvproject{Система анализа пользовательских запросов}
\begin{itemize}
  \item Pipeline для кластеризации и тематического анализа
  \item Визуализация результатов через интерактивные дашборды
  \item Автоматическое выявление ключевых проблем пользователей
\end{itemize}

\divider

\cvproject{Диффузионные модели для генерации контента}
\begin{itemize}
  \item Исследование и применение диффузионных моделей
  \item Генерация изображений по текстовым описаниям
  \item Оптимизация скорости генерации
\end{itemize}

\end{document}
