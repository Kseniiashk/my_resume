\documentclass[margin]{res}
\usepackage{parskip}
\usepackage[utf8]{inputenc}
\usepackage[russian]{babel}
\usepackage{helvet}
\usepackage[dvipsnames]{xcolor}
\usepackage{enumitem}
\usepackage[colorlinks=true, linkcolor=royalblue, urlcolor=royalblue]{hyperref} 

\definecolor{royalblue}{rgb}{0.25, 0.41, 0.88}
\textheight=700pt

\begin{document}

\name{\textcolor{royalblue}{ШКУЛЕВА КСЕНИЯ}}
\address{
    Москва \\
    \href{mailto:ksshkuleva@gmail.com}{ksshkuleva@gmail.com} \\ 
    +7 (965) 178-5997 \\
    GitHub: \href{https://github.com/Kseniiashk}{github.com/Kseniiashk} \\
    Codeforces: \href{https://codeforces.com/profile/KseniaShk}{codeforces.com/profile/KseniaShk} % Исправлено yourusername
}

\begin{resume}

{\color{royalblue} \section{ЦЕЛЬ}}
Специалист в области Data Science, ищущий возможности для применения знаний в разработке интеллектуальных систем. Стремлюсь внести вклад в проекты, связанные с NLP, машинным обучением и RAG-системами.

{\color{royalblue} \section{ОБРАЗОВАНИЕ}}
\textcolor{royalblue}{\textbf{Национальный исследовательский университет <<Высшая школа экономики>> (НИУ ВШЭ)}}\\
{\sl Факультет компьютерных наук, бакалавриат}\\
2024 -- настоящее время\\
Специализация: Компьютерные науки и анализ данных

\textcolor{royalblue}{\textbf{СУНЦ МГУ}}\\
{\sl Школа при Московском государственном университете}\\
2022 -- 2024\\
Специализация: Информатический класс

{\color{royalblue} \section{ОПЫТ РАБОТЫ}}

\vspace{0.5em}
\textbf{\textcolor{royalblue}{IT-компания T-Bank}} \hfill Москва

\textbf{Стажер ML Developer} \hfill Июнь 2024 -- Август 2024
\begin{itemize}[leftmargin=*,topsep=0pt,itemsep=0pt,parsep=0pt]
    \item Автоматизация взаимодействия операторов через Telegram-бота
    \item Разработка методов кластеризации вопросов на основе анализа данных
    \item Деплой решений с использованием Docker
\end{itemize}

\textbf{Стажер RAG Developer} \hfill Март 2024 -- настоящее время
\begin{itemize}[leftmargin=*,topsep=0pt,itemsep=0pt,parsep=0pt]
    \item Разработка RAG-пайплайнов для обработки текстовых данных
    \item Интеграция с чат-ботами через FastAPI
    \item Оптимизация промптов для LLM
\end{itemize}

% Опыт работы в DoorHan
\vspace{0.5em}
\textbf{\textcolor{royalblue}{Группа компаний <<DoorHan>>}} \hfill Москва

\textbf{PHP-программист} \hfill Июнь 2023 -- Март 2024
\begin{itemize}[leftmargin=*,topsep=0pt,itemsep=0pt,parsep=0pt]
    \item Разработка и поддержка веб-приложений
    \item Работа с MySQL, JavaScript, CSS/HTML
\end{itemize}

{\color{royalblue} \section{ТЕХНИЧЕСКИЕ НАВЫКИ}}
\begin{itemize}[leftmargin=*,topsep=0pt,itemsep=0pt,parsep=0pt]
    \item \textbf{Программирование:} Python, PHP, SQL, JavaScript, C++
    \item \textbf{Data Science:} NLP, RAG-системы, кластеризация, TensorFlow
    \item \textbf{Инструменты:} Docker, Git, FastAPI, Jupyter, MySQL
    \item \textbf{Концепции:} Алгоритмы, ML, асинхронное программирование
\end{itemize}

{\color{royalblue} \section{ДОСТИЖЕНИЯ}}
\begin{itemize}[leftmargin=*,topsep=0pt,itemsep=0pt,parsep=0pt]
    \item Победитель Всероссийской олимпиады по информатике (2024)
    \item Призер Всероссийской олимпиады по информатике (2023)
    \item Участник Всероссийской олимпиады по ИИ (2023)
    \item Призер Открытой олимпиады по программированию (2023, 2024)
    \item Призер Всероссийской командной школьной олимпиады по программированию (2022, 2023)
\end{itemize}

{\color{royalblue} \section{ПРОЕКТЫ И ОБУЧЕНИЕ}}
\begin{itemize}[leftmargin=*,topsep=0pt,itemsep=0pt,parsep=0pt]
    \item Telegram-бот для генерации изображений (GAN) и распознования изображений
    \item Запуск и исследование диффузионных моделей
    \item Школа по анализу данных (НИУ ВШЭ)
    \item Курс "Алгоритмы и структуры данных" (Т-Образование, 2020 - 2023)
    \item Летняя школа по разработке на Unreal Engine (НИУ ВШЭ)
\end{itemize}

\end{resume}
\end{document}